\section{ WCA Regulations 2013}

{[}Version: January 1, 2013{]}

\subsection{Notes}

\subsubsection{WCA Guidelines}

The WCA Regulations are supplemented by the . The Regulations should be
considered a complete document, but the Guidelines contain additional
clarifications and explanations.

\subsubsection{Wording}

To make the Regulations and Guidelines easier to read we use ``he''
where the reader should read ``she or he''.\\Uses of the words ``must'',
``must not'', ``should'', ``should not'' and ``may'' match
\href{https://www.ietf.org/rfc/rfc2119.txt}{RFC 2119}.

\subsubsection{Information on the Internet}

Website of World Cube Association:
\href{http://www.worldcubeassociation.org/}{www.worldcubeassociation.org}\\Original
source of the WCA regulations:
\href{http://www.worldcubeassociation.org/regulations/}{www.worldcubeassociation.org/regulations}\\WCA
regulations in \href{wca-regulations-and-guidelines-2013.pdf}{PDF
format}.

\subsubsection{Source}

Development of the WCA Regulations and Guidelines is public
\href{https://github.com/cubing/wca-documents}{on GitHub}.

\subsubsection{Contact}

For questions and feedback, please contact the
\href{http://www.worldcubeassociation.org/contact-information}{WCA
Regulations Committee (WRC)}.

\subsubsection{History}

Past documents are available on the \href{history.html}{history page}

\subsubsection{Translations}

Translations of the 2013 Regulations are not available yet. Links to
older versions can be found in the \href{regulations2010new.html}{2010
Regulations}.

\subsection{Contents}

\subsection{ Article 1: Officials}

\begin{itemize}
\item
  1a) A competition must include a WCA Delegate and an organisation team
  (consisting of one or more individuals) with the following officials:
  judges, scramblers and score takers.
\item
  1b) The organisation team of a competition is responsible for
  logistics before, during, and after the competition.
\item
  1c) The WCA Delegate may delegate responsibilities to other members of
  the organisation team, but is ultimately accountable for how these
  responsibilities are carried out. The WCA Delegate for a competition
  is responsible for:
\item
  1c1) Reporting to the WCA Board regarding adherence to WCA Regulations
  during the competition, the overall course of the competition, and any
  incidents. The report must be submitted to the WCA Board within one
  week of the competition date.
\item
  1c3) Sending the competition results to the WCA Results Team.

  \begin{itemize}
  \item
    1c3a) All scramble sequences used during a competition must be sent
    with the results of the competition.
  \item
    1c3b) Scramble sequences must be labelled with the events, rounds,
    and groups for which they were used.
  \end{itemize}
\item
  1c4) Sending corrections to the competition results to the WCA Board.
\item
  1c5) Advising the other officials where necessary.
\item
  1c6) Approving all events and round formats of a competition, before
  the competition starts, and if changes are required during the
  competition.
\item
  1c7) Decisions about disqualifying competitors during the competition.
\item
  1c8) Providing the scramble sequences.
\item
  1c9) Decisions about changes to the scheduled times of rounds. In
  cases of such changes, a clear announcement must be made to all
  competitors.
\item
  1c10) Making a copy of the WCA Regulations available at the
  competition.
\item
  1e) Each event must have one or more judges.
\item
  1e1) A judge is responsible for executing the procedures of the event.

  \begin{itemize}
  \item
    1e1a) A judge may judge multiple competitors simultaneously at the
    discretion of the Delegate, as long as the judge is able to ensure
    that all WCA Regulations are followed at all times.
  \end{itemize}
\item
  1e2) Capable competitors must be available for judging, if needed by
  organisation team. Penalty: disqualification from the competition.
\item
  1f) Each event must have one or more scramblers. Exception: Fewest
  Moves Solving.
\item
  1f1) A scrambler applies scramble sequences to prepare puzzles for
  attempts.
\item
  1f2) Capable competitors must be available for scrambling, if needed
  by organisation team. Penalty: disqualification from the competition.
\item
  1g) Each event must have one or more score takers.
\item
  1g1) A score taker is responsible for compiling results.
\item
  1g2) Any change to the result on a score sheet must be made at the
  discretion of the WCA Delegate.
\item
  1h) Competitors in the same round of an event may be divided into
  groups.
\item
  1h1) Scramblers and judges for a round should not scramble for/judge
  competitors in their own group before they have finished all of their
  attempts for the round. They may scramble for/judge competitors in
  their own group at the discretion of the WCA Delegate, but the
  organisation team must ensure that scramblers and judges cannot see
  any scrambles for their attempts that they have not attempted yet.
\item
  1j) All officials may compete in the competition.
\item
  1k) Officials may serve multiple roles (e.g.~organisation team, WCA
  Delegate, judge, score taker, scrambler).
\end{itemize}

\subsection{ Article 2: Competitors}

\begin{itemize}
\item
  2a) Any person may compete in a WCA competition if he:
\item
  2a1) Complies with WCA Regulations.
\item
  2a2) Meets the competition requirements, which must be clearly
  announced before the competition.
\item
  2a3) Is not suspended by WCA Board.
\item
  2b) Competitors below the age of 18 must obtain consent from their
  parent(s)/guardian(s) to register and compete.
\item
  2c) Competitors register by providing all information required by the
  organisation team (including: name, country, date of birth, gender,
  contact information, selected events).
\item
  2c1) A competitor is not be eligible to compete without a completed
  registration, as determined by the organisation team.
\item
  2d) A competitor's name, country, gender, and competition results are
  considered public information. All other personal information is
  considered confidential, and must not be disclosed to outside
  organisations/persons without the consent of the competitor.
\item
  2e) Competitors must represent a country of which they hold
  citizenship. The WCA Delegate should verify citizenship standing by
  means of documents (e.g.~a passport). If a competitor is found
  ineligible to represent the country under which they have registered,
  the competitor may be disqualified retroactively and/or suspended, at
  the discretion of the WCA Board.
\item
  2e1) The eligible countries are defined by the Wikipedia article
  ``List of sovereign states'' (``UN member states and observer states''
  and ``Other states'').
\item
  2e2) Competitors with updates to their citizenship status may change
  their country of representation in their first competition of a
  calendar year.
\item
  2f) Competitors must obey venue rules and conduct themselves in a
  considerate manner.
\item
  2g) Competitors must remain quiet when inside the designated
  competition area. Talking is permitted, but must be kept at a
  reasonable level, and away from competitors who are actively
  competing.
\item
  2g3) Competitors in the Competitors Area must not communicate with
  each other about the scrambled states of the puzzles of the round in
  progress. Penalty: disqualification of the competitor from the event,
  at the discretion of the WCA Delegate.
\item
  2h) Competitors must be fully dressed while in the competition venue.
  At the discretion of the Delegate, competitors may be disqualified
  from the competition for inappropriate clothing.
\item
  2i) While competing, competitors must not use electronics or audio
  equipment (e.g.~cell phones, MP3 players, dictaphones, additional
  lighting).
\item
  2i1) Competitors may use certain non-electronic aids that do not give
  the competitor an unfair advantage, at the discretion of the WCA
  Delegate. This includes:

  \begin{itemize}
  \item
    2i1a) Medical/physical aids worn by the competitor (etc. glasses,
    wrist brace, hearing aids).
  \item
    2i1b) Earplugs and earmuffs (but not electronic noise-cancelling
    headphones).
  \end{itemize}
\item
  2j) The WCA Delegate may disqualify a competitor from a specific
  event.
\item
  2j1) If a competitor is disqualified from an event for any reason, he
  is not eligible for any more attempts in the event.
\item
  2j2) If a competitor is disqualified during the course of an event,
  his earlier results remains valid. Exception: cheating or defrauding
  (see ).
\item
  2k) The WCA Delegate may disqualify a competitor from the competition
  (i.e.~from all events of the competition) if the competitor:
\item
  2k1) Fails to check in or register in time for the competition.
\item
  2k2) Is suspected of cheating or defrauding the officials during the
  competition.

  \begin{itemize}
  \item
    2k2a) The WCA Delegate may disqualify any suspected results.
  \end{itemize}
\item
  2k3) Behaves in a way that is unlawful, violent or indecent; or
  intentionally damages venue facilities or personal property within the
  venue.
\item
  2k4) Interferes with, or distracts others during, the competition.
\item
  2k5) Fails to abide by WCA Regulations during the competition.
\item
  2l) A competitor may be disqualified immediately, or after a warning,
  depending on the nature and severity of the infraction.
\item
  2l1) A disqualified competitor is not eligible for the refund of any
  expenses due to participating in the competition.
\item
  2n) Competitors may verbally dispute a ruling to the WCA Delegate.
\item
  2n1) Disputes are only permitted during the competition, within 30
  minutes after the disputed incident happened and before the beginning
  of any following rounds of the relevant event.
\item
  2n2) The WCA Delegate must resolve the dispute before the beginning of
  the next round of the event.
\item
  2n3) The competitor must accept all final rulings of the WCA Delegate.
  Penalty: disqualification from the competition.
\item
  2s) Competitors with other disabilities that may prevent them from
  abiding by one or more WCA Regulations may request special
  accommodations from the WCA Delegate. Competitors requesting such
  accommodations should contact the organisation team and WCA Delegate
  at least two weeks before the competition.
\item
  2t) Each competitor must be familiar with and understand the WCA
  Regulations before the competition.
\end{itemize}

\subsection{ Article 3: Puzzles}

\begin{itemize}
\item
  3a) Competitors must provide their own puzzles for the competition.
\item
  3a1) Competitors must be present and ready to compete when they are
  called to compete for a round. Penalty: disqualification from the
  event.
\item
  3a2) Puzzles must be fully operational, such that normal scrambling is
  possible.
\item
  3a3) Polyhedral puzzles must use a colour scheme with one unique
  colour per face in the solved state. Each puzzle variation must have
  moves, states, and solutions functionally identical to the original
  puzzle.
\item
  3d) Puzzles must have coloured stickers, coloured tiles, or
  painted/printed colours.
\item
  3d1) Exception: Competitors with a medically documented visual
  disability may use textured puzzles with different textures on
  different faces. Textures/patterns must be uniform per face. Each face
  should have a distinct colour, to aid in scrambling and judging.
\item
  3d2) The colours of puzzles must be solid, with one uniform colour per
  face. Each colour on the puzzle must be clearly distinct from the
  other colours.
\item
  3d3) Stickers/tiles/textures/paint must not be thicker than 1.5 mm, or
  the generally available thickness for non-cube puzzles.
\item
  3h) Modifications that enhance the basic concept of a puzzle are not
  permitted. Modified versions of puzzles are permitted only if the
  modification does not make any additional information available to the
  competitor (e.g.~identity of pieces), compared to an unmodified
  version of the same puzzle.
\item
  3h1) ``Pillowed'' puzzles are not permitted. Exception: Pillowed 7x7x7
  cubes are permitted.
\item
  3h2) ``Stickerless'' cubes, and other cubes whose face colours are
  visible inside the cube, are not permitted.
\item
  3h3) Any modifications to a puzzle that result in poor performance by
  a competitor are not grounds for additional attempts.
\item
  3j) Puzzles must be clean, and must not have any markings, elevated
  pieces, damage, or other differences that distinguish any piece from a
  similar piece.
\item
  3j1) Puzzles are permitted to have reasonable wear, at the discretion
  of the WCA Delegate.
\item
  3k) Puzzles must be approved by the WCA Delegate before use in the
  competition.
\item
  3l) A cube puzzle may have a logo, but it must have at most one logo.
  For the Rubik's Cube or bigger cube puzzles the logo must be placed on
  one of the centre pieces.
\item
  3l1) Colourless engravings (max. 1 per face) are not considered logos.
\item
  3m) All brands of puzzles and puzzle parts are permitted, as long as
  the puzzles comply with all WCA Regulations.
\end{itemize}

\subsection{ Article 4: Scrambling}

\begin{itemize}
\item
  4a) A scrambler applies scramble sequences to the puzzles.
\item
  4b) Puzzles must be scrambled using computer-generated random scramble
  sequences.
\item
  4b1) Generated scramble sequences must not be inspected before the
  competition, and must not be filtered or selected in any way by the
  WCA Delegate.
\item
  4b2) Scramble sequences for a round must be available only to the WCA
  Delegate and the scramblers for the event, until the end of the round.
  Exception: For Fewest Moves Solving, competitors receive scrambling
  sequences during the round (see ).
\item
  4d) Scrambling orientation:
\item
  4d1) Cube puzzles and Megaminx are scrambled with the white face (if
  not possible, then the lightest face) on top and the green face (if
  not possible, then the darkest adjacent face) on the front.
\item
  4d2) Pyraminx are scrambled with the yellow face (if not possible,
  then the lightest face) on bottom and the green face (if not possible,
  then the darkest adjacent face) on the front.
\item
  4d3) Square-1 are scrambled with the darker colour on front (out of
  the 2 possible scrambling orientations).
\item
  4f) Competition scramble sequences must be generated using the current
  official version of the official WCA scramble program (available via
  the WCA website).
\item
  4g) After scrambling a puzzle, the scrambler must verify that he has
  scrambled the puzzle correctly. If the puzzle state is wrong, he must
  correct it (e.g.~by solving the puzzle and applying the scramble
  sequence again).
\item
  4g1) Exception: For the 6x6x6 Cube and the 7x7x7 Cube, it is not
  necessary to correct the scramble, at the discretion of the WCA
  Delegate.
\end{itemize}

\subsection{ Article 5: Puzzle Defects}

\begin{itemize}
\item
  5a) Examples of puzzle defects include: popped pieces, pieces twisted
  in place, and detached screws/caps/stickers.
\item
  5b) If a puzzle defect occurs during an attempt, the competitor may
  choose to either repair the defect and continue the attempt, or to
  stop the attempt.
\item
  5b1) If a competitor chooses to repair the puzzle, he must repair only
  the defective pieces. Tools and/or pieces of other puzzles must not be
  used to repair the original puzzle. Penalty: disqualification of the
  attempt (DNF).
\item
  5b2) Any repair to a puzzle must not give the competitor any advantage
  in solving the puzzle. Penalty: disqualification of the attempt (DNF).
\item
  5b3) Permitted repairs:

  \begin{itemize}
  \item
    5b3a) If any pieces have fallen out or moved out of place, the
    competitor may place them back.
  \item
    5b3b) If, after repairing the puzzle but before the end of the
    attempt, the competitor finds that the puzzle is unsolvable, he may
    disassemble and reassemble a maximum of 4 pieces to make the puzzle
    solvable.
  \item
    5b3c) If the puzzle is unsolvable, and can be made solvable by
    rotating a single corner piece, the competitor may correct the
    corner piece by twisting it in place without disassembling the
    puzzle.
  \end{itemize}
\item
  5b4) During blindfolded events, a puzzle defect must be repaired
  during the attempt, and all repairs must be performed blindfolded.
  Penalty: disqualification of the attempt (DNF).
\item
  5b5) If parts of the puzzle are still defective or not fully placed at
  the end of the attempt, the result is recorded as the worst state
  obtainable by repairing the puzzle (see ).
\item
  5c) If a competitor has a puzzle defect, this does not give him the
  right to an extra attempt.
\end{itemize}

\subsection{ Article 7: Environment}

\begin{itemize}
\item
  7b) Spectators must remain at least 1.5 metres away from the solving
  stations when they are in use.
\item
  7c) Lighting of the competition area must be given special attention.
  Lighting should be neutral, such that competitors can easily
  differentiate among the colours on the puzzles.
\item
  7e) The competition area must be smoke-free.
\item
  7f) Solving station:
\item
  7f1) Definitions:

  \begin{itemize}
  \item
    7f1a) Stackmat: The Speedstacks Stackmat timer and a full-size
    compatible mat.
  \item
    7f1b) Mat: The mat of the Stackmat.
  \item
    7f1c) Timer: The timer of the Stackmat.
  \item
    7f1d) Surface: The flat surface on which the Stackmat has been
    placed. The mat is considered a part of the surface. The timer is
    not considered a part of the surface.
  \end{itemize}
\item
  7f2) The timer must be attached to the mat and placed on the surface,
  with the timer on the side of the mat nearest to the competitor.

  \begin{itemize}
  \item
    7f2a) Exception: For Solving With Feet, the Stackmat must be placed
    directly on the floor. The timer device may be placed on the side of
    the mat farthest from the competitor.
  \end{itemize}
\item
  7h) The competition area must have a Competitors Area.
\item
  7h1) The organisation team may require that a competitor who has been
  called to compete must remain within the Competitors Area until he has
  finished all of his attempts for the round.
\end{itemize}

\subsection{ Article 8: Competitions}

\begin{itemize}
\item
  8a) An official WCA competition must:
\item
  8a1) Be approved by the WCA Board.
\item
  8a2) Follow the WCA Regulations.
\item
  8a3) Have a designated WCA Delegate in attendance.
\item
  8a4) Be announced on the WCA website at least two weeks before the
  beginning of the competition.
\item
  8a6) Be publicly accessible.
\item
  8a7) Use the authentic Speed Stacks Stackmat timer (Generation 2 or
  Pro) for time measurement.
\item
  8a8) Be open to all who wish to compete. Restrictions must be approved
  by the WCA Board and clearly stated when the competition is announced.
\item
  8f) If WCA Regulations are not correctly observed during a
  competition, the WCA Board may disqualify affected attempts.
\end{itemize}

\subsection{ Article 9: Events}

\begin{itemize}
\item
  9a) The WCA governs competitions for:
\item
  9a1) Puzzles known as Rubik's puzzles.
\item
  9a2) Other puzzles that are manipulated by twisting the sides,
  commonly known as ``twisty puzzles''.
\item
  9b) The official puzzles and event formats of the WCA are:
\item
  9b1) Rubik's Cube, 2x2x2 Cube, 4x4x4 Cube, 5x5x5 Cube, Clock,
  Megaminx, Pyraminx, Square-1, and Rubik's Cube: One-Handed.

  \begin{itemize}
  \item
    9b1a) Competition formats for these events are: ``Best of X'' (where
    X is 1, 2, or 3), and ``Average of 5''.
  \end{itemize}
\item
  9b2) Rubik's Cube: With Feet, 6x6x6 Cube, and 7x7x7 Cube.

  \begin{itemize}
  \item
    9b2a) Competition formats for these events are: ``Best of X'' (where
    X is 1, 2, or 3) and ``Mean of 3''.
  \end{itemize}
\item
  9b3) Rubik's Cube: Fewest Moves, Rubik's Cube: Blindfolded, 4x4x4
  Cube: Blindfolded, 5x5x5 Cube: Blindfolded, and Rubik's Cube: Multiple
  Blindfolded.

  \begin{itemize}
  \item
    9b3a) Competition formats for these events are: ``Best of X'' (where
    X is 1, 2, or 3).
  \end{itemize}
\item
  9f) The results of a round are measured as follows:
\item
  9f1) All timed results under 10 minutes are measured and rounded down
  to the nearest hundredth of a second. All timed averages and means
  under 10 minutes are measured and rounded to the nearest hundredth of
  a second.
\item
  9f2) All timed results, averages, and means over 10 minutes are
  measured and rounded to the nearest second (e.g.~x.4 becomes x, x.5
  becomes x+1).
\item
  9f4) The result of an attempt is recorded as DNF (Did Not Finish) if
  the attempt is disqualified or unsolved/unfinished.
\item
  9f5) The result of an attempt is recorded as DNS (Did Not Start) if
  the competitor is eligible for an attempt but declines it.
\item
  9f6) For ``Best of X'' rounds, each competitor is allotted X attempts.
  The best result of these attempts counts for the competitor's ranking
  in the round.
\item
  9f7) For ``Best of X'' rounds, a DNF or DNS is the worst possible
  result.
\item
  9f8) For ``Average of 5'' rounds, competitors are allotted 5 attempts.
  Of these 5 attempts, the best and worst attempts is removed, and the
  arithmetic mean of the remaining 3 attempts counts for the
  competitor's ranking in the round.
\item
  9f9) For ``Average of 5'' rounds, one DNF or DNS is permitted to count
  as the competitor's worst result of the round. If a competitor has
  more than one DNF and/or DNS result in the round, his average score
  for the round is DNF.
\item
  9f10) For ``Mean of 3'' rounds competitors are allotted 3 attempts.
  The arithmetic mean of the 3 attempts counts for the competitor's
  ranking in the round.
\item
  9f11) For ``Mean of 3'' rounds, if the competitor has at least one DNF
  or DNS result, his average score for the round is DNF.
\item
  9f12) For ``Best of X'' rounds, rankings are assessed based on the
  best result per competitor. The following are used to compare results:

  \begin{itemize}
  \item
    9f12a) For timed results, ``better'' is defined as the shorter time.
  \item
    9f12b) For Fewest Moves Solving, ``better'' is defined as the
    shorter solution length.
  \item
    9f12c) For Multiple Blindfolded Solving, rankings are assessed based
    on number of puzzles solved minus the number of puzzles not solved,
    where a greater difference is better. If the difference is less than
    0, the attempt is considered unsolved (DNF). If competitors achieve
    the same result, rankings are assessed based on total time, where
    the shorter recorded time is better. If competitors achieve the same
    result and the same time, rankings are assessed based on the number
    of puzzles the competitors failed to solve, where fewer unsolved
    puzzles is better.
  \end{itemize}
\item
  9f13) For ``Mean of 3'' and ``Average of 5'' rounds, rankings are
  assessed based on the ordering of the averages/means of the
  competitors, where ``better'' is the shorter recorded time.
\item
  9f14) For ``Mean of 3'' and ``Average of 5'' rounds, if two or more
  competitors achieve identical average/mean results, rankings are
  assessed based on the best attempt per competitor, where ``better'' is
  defined as the shorter time.
\item
  9f15) Competitors who achieve the same result in a round receive an
  identical ranking for the round.
\item
  9g) A Combined Round consists of two phases of attempts, where
  competitors advance to the second phase if they meet a designated
  cutoff during the first phase.
\item
  9g2) Whether a competitor proceeds to next phase of a Combined round,
  must be decided by ranking (best x competitors) or by result (all
  competitors with a best result under x) of the first phase.
\item
  9i) Results of official WCA competitions must be listed on the WCA
  world rankings.
\item
  9i1) The WCA recognises the following types of regional records:
  national records, continental records, and world records.
\item
  9i2) All the results of a round are considered to take place on the
  last calendar date of the round. If a regional record is broken
  multiple times on the same calendar date, only the best result is
  recognised as breaking that regional record.
\item
  9i3) If the WCA Regulations for an event are changed, existing
  regional records stand until they are broken under the new WCA
  Regulations.
\item
  9j) Each event must be held at most once per competition.
\item
  9k) All competitors may participate in all events of a competition,
  except in cases specifically approved by the Board.
\item
  9l) Each round must be completed before any following round of the
  same event starts.
\item
  9m) Events must have at most four rounds.
\item
  9m1) Events with 99 or fewer competitors must have at most three
  rounds.
\item
  9m2) Events with 15 or fewer competitors must have at most two rounds.
\item
  9m3) Events with 7 or fewer competitors must have at most one round.
\item
  9o) Combined rounds and Qualification rounds each count as one round
  when counting the number of rounds per event.
\item
  9p) If an event has multiple rounds, then:
\item
  9p1) At least 25\% of competitors must be eliminated between
  consecutive rounds of the same event.
\item
  9p2) The competitors who advance to the next round must be determined
  by either a cutoff ranking or a cutoff time in the preceding
  elimination round.
\item
  9p3) If a qualifying competitor withdraws from a round, he may be
  replaced by the best-ranked competitor below the cutoff from the
  preceding round.
\item
  9r) A qualification round must be held before the first round of the
  event.
\item
  9r1) When announcing an event, the organisation team must include:

  \begin{itemize}
  \item
    9r1a) Any limit to the number of competitors permitted in the first
    round of the event.
  \item
    9r1b) Any limit to the number of competitors permitted in the
    qualification round of the event, and any limit to the maximum
    number of those competitors who will proceed to the first round.
  \item
    9r1c) The average/mean result or single solve result, achieved in a
    previous competition, required to qualify directly for the first
    round of the event.
  \item
    9r1d) The latest permitted date used to determine the qualifying
    average/mean result or single solve result in the WCA rankings.
  \end{itemize}
\item
  9r2) Each competitor of an event who has not directly qualified for
  the first round of the event must compete in the qualification round
  in order to participate in the first round.
\item
  9r3) A qualification round may be added to specifically accommodate
  newly registered competitors, and/or the maximum number of competitors
  in the qualification round or first round of an event may be
  increased. These changes must be made at least two weeks before the
  competition.
\item
  9s) Each round of each event must have a time limit (see ).
\end{itemize}

\subsection{ Article 10: Solved State}

\begin{itemize}
\item
  10b) Only the resting state of the puzzle, after the timer has been
  stopped, is considered.
\item
  10c) The puzzle may be in any orientation at the end of the attempt.
\item
  10d) All pieces of a puzzle must be fully attached to the puzzle in
  their required positions. Exception: see .
\item
  10e) A puzzle is solved when all face colours are reassembled and all
  parts are aligned within the limits specified below:
\item
  10e1) For each two adjacent parts (e.g.~two parallel, adjacent slices
  of a cube) of the puzzle that are misaligned more than the limit
  described in , the puzzle is considered to require one additional move
  to solve (see ``Outer Block Turn Metric'' in ).
\item
  10e2) If no further moves are required to bring the puzzle to its
  solved state, the puzzle is considered solved without penalty.
\item
  10e3) If one move is required, the puzzle is be considered solved with
  a time penalty (+2 seconds).
\item
  10e4) If more than one move is required, the puzzle is considered
  unsolved (DNF).
\item
  10f) Limits of acceptable misalignment for puzzles:
\item
  10f1) Cube-shaped puzzles: at most 45 degrees.
\item
  10f2) Megaminx: at most 36 degrees.
\item
  10f3) Pyraminx: at most 60 degrees.
\item
  10f4) Square-1: at most 45 degrees (U/D) or 90 degrees (/).
\item
  10h) Puzzles not specified in this article are judged according to the
  solved state as defined by the generally accepted goal of the puzzle,
  applying the relevant regulations for the Rubik's Cube.
\end{itemize}

\subsection{ Article 11: Incidents}

\begin{itemize}
\item
  11a) Incidents include:
\item
  11a1) Incorrect execution of event procedures, by officials or
  competitors.
\item
  11a2) Interference or facility interruptions (e.g.~power failure,
  emergency alarm activation).
\item
  11a3) Equipment malfunction.
\item
  11b) If an incident occurs, the WCA Delegate determines an impartial
  and appropriate course of action.
\item
  11d) If the WCA Regulations are not fully clear or if the incident is
  not covered by the WCA Regulations, then the WCA Delegate must make
  his decision based on fair sportsmanship.
\item
  11e) If an incident occurs during an attempt, the WCA Delegate may
  award a competitor an extra attempt, replacing the attempt during
  which the incident occurred. The competitor must appeal verbally or in
  writing to the judge and WCA Delegate at the time of the incident,
  before finishing the original attempt, to be eligible for an extra
  attempt. An appeal does not guarantee the competitor an extra attempt.
\item
  11e1) If a competitor is awarded an extra attempt, the extra attempt
  must be scrambled using a different scramble sequence. This scramble
  sequence must be generated using the current official version of the
  official WCA scramble program (see ).
\item
  11f) Decisions about an incident may be supported with video or
  photographic analysis, at the discretion of the WCA Delegate.
\end{itemize}

\subsection{ Article 12: Notation}

\begin{itemize}
\item
  12a) Notation for Rubik's Cube and similar puzzles:
\item
  12a1) Face Moves:

  \begin{itemize}
  \item
    12a1a) Clockwise, 90 degrees: F (front face), B (back face), R
    (right face), L (left face), U (upper face), D (bottom face).
  \item
    12a1b) Anti-clockwise, 90 degrees: F', B', R', L', U', D' (see ).
  \item
    12a1c) 180 degrees: F2, B2, R2, L2, U2, D2 (see ).
  \end{itemize}
\item
  12a2) Multiple Outer Slice Moves (outer slice plus adjacent inner
  slices; n is defined as total number of slices to move; n may be
  omitted for two slices):

  \begin{itemize}
  \item
    12a2a) Clockwise, 90 degrees: nFw, nBw, nRw, nLw, nUw, nDw. (see ).
  \item
    12a2b) Anti-clockwise, 90 degrees: nFw', nBw', nRw', nLw', nUw',
    nDw' (see ).
  \item
    12a2c) 180 degrees: nFw2, nBw2, nRw2, nLw2, nUw2, nDw2 (see ).
  \end{itemize}
\item
  12a3) Outer Block Turn Metric (OBTM) is defined as:

  \begin{itemize}
  \item
    12a3a) Each move of the categories Face Moves and Multiple Outer
    Slice Moves is counted as 1 move.
  \item
    12a3b) Each rotation is counted as 0 moves.
  \end{itemize}
\item
  12b) Rotations for all cube shaped puzzles:
\item
  12b1) Clockwise, 90 degrees: {[}f{]} or z, {[}b{]} or z', {[}r{]} or
  x, {[}l{]} or x', {[}u{]} or y, {[}d{]} or y'. (see ).
\item
  12b2) Anti-clockwise, 90 degrees: {[}f'{]} or z', {[}b'{]} or z,
  {[}r'{]} or x', {[}l'{]} or x, {[}u'{]} or y', {[}d'{]} or y (see ).
\item
  12b3) 180 degrees: {[}f2{]} or z2, {[}b2{]} or z2, {[}r2{]} or x2,
  {[}l2{]} or x2, {[}u2{]} or y2, {[}d2{]} or y2 (see ).
\item
  12c) Notation for Square-1:
\item
  12c1) Moves are applied with the smallest slice of the middle layer on
  left side of front face.
\item
  12c2) (x,y) means: turn upper layer x times 30 degrees clockwise, turn
  bottom layer y times 30 degrees clockwise.
\item
  12c3) ``/'' means: turn the right half of the puzzle 180 degrees.
\item
  12d) Notation for Megaminx (scrambling notation only):

  \begin{itemize}
  \item
    12d1a) Clockwise, 72 degrees: U (upper face).
  \item
    12d1b) Anti-clockwise, 72 degrees: U' (upper face).
  \end{itemize}
\item
  12d2) Other moves are applied while keeping 3 pieces fixed at top left
  of the puzzle:

  \begin{itemize}
  \item
    12d2c) Clockwise 144 degrees move of whole puzzle except for the
    slice of top left three pieces: R++ (vertical slices), D++
    (horizontal slices).
  \item
    12d2d) Anti-clockwise 144 degrees move of whole puzzle except for
    the slice of top left three pieces: R-- (vertical slices), D--
    (horizontal slices).
  \end{itemize}
\item
  12e) Notation for Pyraminx:
\item
  12e1) The puzzle is oriented with the bottom face completely
  horizontal and the front face facing the person who holds the
  Pyraminx.
\item
  12e2) Clockwise, 120 degrees: U (upper 2 layers), L (left 2 layers), R
  (right 2 layers), B (back 2 layers), u (upper vertex), l (left
  vertex), r (right vertex), b (back vertex).
\item
  12e3) Anti-clockwise, 120 degrees: U' (upper 2 layers), L' (left 2
  layers), R' (right 2 layers), B' (back 2 layers), u' (upper vertex),
  l' (left vertex), r' (right vertex), b' (back vertex).
\item
  12g) Notation for Clock:
\item
  12g1) The puzzle is oriented with twelve on top, and either side in
  front.
\item
  12g2) Move pins up: UR (top-right), DR (bottom-right), DL
  (bottom-left), UL (top-left), U (both top), R (both right), D (both
  bottom), L (both left), ALL (all).
\item
  12g3) Turn a wheel next to an up-position pin and move all pins down
  afterwards: x+ (x clockwise turns), x- (x anti-clockwise turns).
\item
  12g4) Turn around the puzzle so that twelve is still on top, and then
  move all pins down: y2.
\end{itemize}

\subsection{ Article A: Speed Solving}

\begin{itemize}
\item
  A1) Speed Solving attempts must abide by the following procedure.
\item
  A1a) The organisation team may enforce time limits for attempts and/or
  rounds.

  \begin{itemize}
  \item
    A1a1) The default time limit per attempt is 10 minutes, though the
    organisation team may announce a higher or lower time limit.
  \item
    A1a2) Cumulative time limits may be enforced (e.g.~3 attempts with a
    cumulative time limit of 20 minutes). The time elapsed in a DNF
    result counts towards the cumulative time limit.
  \item
    A1a3) For each round, any time limits must be announced before the
    round starts, and should not be changed after it has begun. Changes
    must be made at the discretion of the WCA Delegate, who must
    carefully consider the fairness of the change.
  \item
    A1a4) The competitor must end each attempt within the time limit. If
    a competitor reaches the time limit for an attempt/round, the judge
    stops the attempt immediately and record the result as DNF.
    Exception: Multiple Blindfolded Solving (see ).
  \item
    A1a5) An attempt is considered to meet the time limit if and only if
    the final result, after any time penalties are applied, is less than
    the time limit. Exception: Multiple Blindfolded Solving (see ).
  \end{itemize}
\item
  A1b) If the time limit for an attempt is greater than 10 minutes, a
  stopwatch must be used for timekeeping.

  \begin{itemize}
  \item
    A1b1) Simultaneous use of a Stackmat timer is optional.
  \item
    A1b2) If a time from the Stackmat timer is available, it is the
    original recorded time. Otherwise, the stopwatch time is the
    original recorded time.
  \end{itemize}
\item
  A1c) A competitor participating in an event must be able to fulfill
  the event's requirements (e.g.~know how to solve the puzzle). A
  competitor competing with expectation of a DNF result may be
  disqualified from the event, at the discretion of the WCA Delegate.
\item
  A2) Scrambling:
\item
  A2a) When called for a round, the competitor submits his puzzle, in
  its solved state, to the scrambler and waits in the Competitors Area
  until he is called to compete.
\item
  A2b) A scrambler scrambles the puzzle according to the regulations in
  .
\item
  A2c) After the scrambler starts scrambling the puzzle, the competitor
  must not see the puzzle until the inspection phase starts.

  \begin{itemize}
  \item
    A2c1) The scrambler places a cover over the scrambled puzzle that
    makes it impossible for any competitors or spectators to see any
    part of the puzzle. The cover remains over the puzzle until the
    beginning of the attempt.
  \end{itemize}
\item
  A2d) When taking a puzzle from the scrambler, the judge briefly
  inspects the puzzle to ensure thorough scrambling of the puzzle. The
  judge raises any concerns with the scrambler, who then conducts a
  detailed check.
\item
  A2e) The judge places the puzzle onto the mat in an arbitrary
  orientation and ensures that it is covered completely. The competitor
  is not permitted to request a specific orientation.
\item
  A3) Inspection:
\item
  A3a) The competitor may inspect the puzzle at the beginning of each
  attempt.

  \begin{itemize}
  \item
    A3a1) The competitor is allotted a maximum of 15 seconds to inspect
    the puzzle and start the solve.
  \end{itemize}
\item
  A3b) Before the competitor starts the attempt, the judge resets the
  timer and, where applicable, the stopwatch.

  \begin{itemize}
  \item
    A3b1) When the judge believes the competitor is ready, he asks
    ``READY?''. The competitor must be ready to start the attempt within
    one minute of being called. Penalty: disqualification of the attempt
    (DNF), at the discretion of the judge.
  \item
    A3b2) When the competitor confirms his readiness, the judge uncovers
    the puzzle. If the attempt requires a stopwatch, the judge starts it
    at the same time.
  \end{itemize}
\item
  A3c) The competitor may pick up the puzzle during inspection.

  \begin{itemize}
  \item
    A3c1) The competitor must not apply moves during inspection.
    Penalty: disqualification of the attempt (DNF).
  \item
    A3c2) If the pieces of the puzzle are not fully aligned, then the
    competitor may align the faces, as long as misalignments stay within
    the limits 10f.
  \item
    A3c3) The competitor may reset the timer before he starts the solve.
  \end{itemize}
\item
  A3d) At the end of the inspection, the competitor places the puzzle on
  the mat, in any orientation and position.

  \begin{itemize}
  \item
    A3d1) The puzzle must not rest on the timer. Penalty: time penalty
    (+2 seconds).
  \item
    A3d2) When 8 seconds of inspection have elapsed, the judge calls ``8
    SECONDS''.
  \item
    A3d3) When 12 seconds of inspection have elapsed, the judge calls
    ``12 SECONDS''.
  \end{itemize}
\item
  A4) Starting the solve:
\item
  A4b) The competitor places his hands on the elevated sensor unit of
  the timer, with his fingers touching the sensors and palms down.
  Penalty: time penalty (+2 seconds).

  \begin{itemize}
  \item
    A4b1) The competitor must have no physical contact with the puzzle
    between the inspection period and the beginning of the solve.
    Penalty: time penalty (+2 seconds).
  \end{itemize}
\item
  A4d) The competitor starts the solve by confirming that the timer
  light is green and then removing his hands from the timer, thus
  starting the timer.

  \begin{itemize}
  \item
    A4d1) The competitor must start the solve within 15 seconds of the
    beginning of the inspection. Penalty: time penalty (+2 seconds).
  \item
    A4d2) The competitor must start the solve within 17 seconds of the
    beginning of the inspection. Penalty: disqualification of the
    attempt (DNF).
  \end{itemize}
\item
  A4e) Time penalties for starting the solve are cumulative.
\item
  A5) During the solve:
\item
  A5a) While inspecting or solving the puzzle, the competitor must not
  communicate with anyone other than the judge. Penalty:
  disqualification of the attempt (DNF).
\item
  A5b) While inspecting or solving the puzzle, the competitor must not
  receive assistance from anyone or any object other than the surface
  (also see ). Penalty: disqualification of the attempt (DNF).
\item
  A6) Stopping the solve:
\item
  A6a) The competitor stops the solve by releasing the puzzle and then
  stopping the timer.

  \begin{itemize}
  \item
    A6a1) If a stopwatch is in use as the timing device, the competitor
    ends the solve by releasing the puzzle and notifying the judge that
    he has stopped the solve.
  \item
    A6a2) When using a stopwatch without a Stackmat, the competitor's
    default notification signal consists of releasing the puzzle(s) in
    his hand and placing his hands on the surface, with palms down. The
    competitor and the judge may agree on another appropriate
    notification before the beginning of the solve.
  \end{itemize}
\item
  A6b) The competitor is responsible for stopping the timer correctly.

  \begin{itemize}
  \item
    A6b1) If the timer stops before the end of the solve and the timer
    shows a time strictly below 0.06 seconds, then the attempt is
    replaced by an extra attempt. A competitor forfeits his right to the
    additional attempt if the WCA Delegate determines that the timer was
    stopped deliberately.
  \item
    A6b2) If the timer stops before the end of the solve and displays a
    time of 0.06 seconds or higher, then the attempt is disqualified
    (DNF). Exception: if the competitor can demonstrate that the timer
    malfunctioned, he may receive an extra attempt, at the WCA
    Delegate's discretion.
  \end{itemize}
\item
  A6c) The competitor must fully release the puzzle before stopping the
  timer. Penalty: time penalty (+2 seconds).
\item
  A6d) The competitor must stop the timer using both hands, placed flat
  on the sensors with palms down. Penalty: time penalty (+2 seconds).
\item
  A6e) The competitor must not touch or move the puzzle until the judge
  has inspected the puzzle. Penalty: disqualification of the attempt
  (DNF). Exception: If no moves have been applied, a time penalty (+2
  seconds) may be assigned instead, at the discretion of the judge.
\item
  A6f) The competitor must not reset the timer until the judge has
  recorded the result on the score sheet. Penalty: disqualification of
  the attempt (DNF), at the discretion of the judge.
\item
  A6g) The judge determines whether the puzzle is solved. He must not
  make moves or align faces when examining the puzzle. Exception: The
  judge is permitted to make moves when examining a Clock.
\item
  A6h) In case of a dispute, moves or alignments must not be applied to
  the puzzle before the dispute is resolved.
\item
  A6i) Time penalties for stopping the solve are cumulative.
\item
  A7) Recording results:
\item
  A7a) The judge tells the competitor the result.

  \begin{itemize}
  \item
    A7a1) If the judge finds that the puzzle is solved, he calls
    ``OKAY''.
  \item
    A7a2) If the judge assigns any penalties, the judge calls
    ``PENALTY''.
  \item
    A7a3) If the result is DNF, the judge calls ``DNF''.
  \end{itemize}
\item
  A7b) The judge records the result on a score sheet.

  \begin{itemize}
  \item
    A7b1) If penalties are assigned, the judge records the original
    recorded result displayed on the timer, along with any penalties.
    The format should be ``X + T + Y = F'', where X represents the sum
    of time penalties before/starting the solve, T represents the time
    displayed on the timer, Y represents a sum of time penalties
    during/after the solve, and F represents the final result. If X
    and/or Y is 0, the 0 terms are omitted (e.g.~2 + 17.65 + 2 = 21.65,
    or 17.65 + 2 = 19.65).
  \end{itemize}
\item
  A7c) The judge and competitor must both sign (or initial) the score
  sheet to acknowledge the result.

  \begin{itemize}
  \item
    A7c1) If the competitor or the judge refuses to accept and sign the
    score sheet, the WCA Delegate must resolve the dispute.
  \end{itemize}
\item
  A7f) When a competitor's score sheet for a round is complete, the
  judge delivers the score sheet to the score taker.
\end{itemize}

\subsection{ Article B: Blindfolded Solving}

\begin{itemize}
\item
  B1) Standard speed solving procedures is followed, as described in
  (Speed Solving). Additional regulations that supersede the
  corresponding procedures in are described below.
\item
  B1a) There is no inspection period.
\item
  B1b) The competitor supplies his own blindfold.
\item
  B2) Starting the attempt:
\item
  B2a) The judge resets the timer and stopwatch.
\item
  B2b) The competitor places his hands on the elevated sensor unit of
  the Stackmat, with his fingers touching the sensors and palms down.
  Penalty: time penalty (+2 seconds).
\item
  B2c) The competitor must have no physical contact with the puzzle
  before the beginning of the attempt. Penalty: time penalty (+2
  seconds).
\item
  B2d) The competitor starts the attempt by removing his hands from the
  timer, thus starting the timer.

  \begin{itemize}
  \item
    B2d1) The competitor removes the cover from the puzzle after
    starting the timer.
  \end{itemize}
\item
  B2e) If a stopwatch is in use, the judge starts the stopwatch as soon
  as the competitor starts the solve.
\item
  B3) Memorisation phase:
\item
  B3a) The competitor may pick up the puzzle during the memorisation
  phase.
\item
  B3b) The competitor must not make physical notes. Penalty:
  disqualification of the attempt (DNF).
\item
  B4b) The competitor must not apply moves to the puzzle during the
  memorisation phase.
\item
  B4) Blindfolded phase:
\item
  B4a) The competitor dons the blindfold to start the blindfolded phase.
\item
  B4b) The competitor must not apply moves to the puzzle before he has
  fully donned the blindfold.
\item
  B4c) The judge must ensure that there is an opaque object between the
  competitor's face and the puzzle while the competitor is solving.

  \begin{itemize}
  \item
    B4c1) In all cases, the competitor must wear the blindfold such that
    his view of the puzzle would still be clearly blocked if the opaque
    object were not in the way.
  \item
    B4c2) By default, the judge should place the object (e.g.~a sheet of
    paper or cardboard) between the competitor and the puzzle while the
    competitor is wearing the blindfold.
  \item
    B4c3) If the judge and competitor agree beforehand, the competitor
    may choose to place the puzzle behind a suitable object (e.g.~a
    music stand, the surface of the table) by himself during the
    blindfolded phase.
  \end{itemize}
\item
  B4d) The competitor must not look at the puzzle at any point during
  the blindfolded phase. Penalty: disqualification of the attempt (DNF).
\item
  B4e) Until he applies the first move to the puzzle, the competitor may
  remove the blindfold to return to the memorisation phase.
\item
  B5) Stopping the solve:
\item
  B5a) When using the Stackmat, the competitor stops the attempt by
  releasing the puzzle and then stopping the timer.
\item
  B5b) When using a stopwatch, the competitor ends the attempt by
  placing the puzzle back onto the surface and notifying the judge that
  he is stopping the attempt. At that moment, the judge stops the timer.
\item
  B5c) If he is not touching the puzzle, the competitor may remove the
  blindfold before he stops the timer. He must not touch the puzzle
  until the end of the attempt. Penalty: disqualification of the attempt
  (DNF).
\end{itemize}

\subsection{ Article C: One-Handed Solving}

\begin{itemize}
\item
  C1) Standard speed solving procedures is followed, as described in
  (Speed Solving). Additional regulations that supersede the
  corresponding procedures in are described below.
\item
  C1b) During the solve, the competitor must use only one hand to touch
  the puzzle. Penalty: disqualification of the attempt (DNF).

  \begin{itemize}
  \item
    C1b2) If a puzzle defect occurs, and the competitor chooses to
    repair it, he must repair it using only the solving hand. Penalty:
    disqualification of the attempt (DNF).
  \item
    C1b3) If a puzzle defect occurs, and pieces of the puzzle briefly
    come in contact with other body parts without the competitor's
    intention, this is not considered touching the puzzle, at the
    discretion of the judge.
  \end{itemize}
\item
  C1c) During the solve, once a competitor touches the puzzle with one
  hand, he must not touch the puzzle with the other hand. Penalty:
  disqualification of the attempt (DNF).
\end{itemize}

\subsection{ Article D: Solving With Feet}

\begin{itemize}
\item
  D1) Standard speed solving procedures is followed, as described in
  (Speed Solving). Additional regulations that supersede the
  corresponding procedures in are described below.
\item
  D1a) During the attempt, the competitor must sit in a chair, sit on
  the surface, or stand.
\item
  D1b) During the attempt, the competitor must only use his feet and the
  surface. Penalty: disqualification of the attempt (DNF).
\item
  D1c) During the solve, the competitor must use only his feet to touch
  the puzzle. Penalty: disqualification of the attempt (DNF).
\item
  D3) Starting the solve:
\item
  D3a) The competitor places his feet onto the timer sensors.
\item
  D3b) The competitor removes his feet from the timer sensors to start
  the solve.
\item
  D4) Stopping the solve:
\item
  D4a) The competitor stops the timer by placing his feet onto the timer
  sensors.
\end{itemize}

\subsection{ Article E: Fewest Moves Solving}

\begin{itemize}
\item
  E2) Procedure for Fewest Moves Solving:
\item
  E2a) The judge distributes a scramble sequence to all competitors. The
  judge then starts the stopwatch and calls ``GO''.
\item
  E2b) All competitors have a total time limit of 60 minutes to devise a
  solution.

  \begin{itemize}
  \item
    E2b1) The judge should call ``5 MINUTES REMAINING'' at 55 minutes,
    and must call ``STOP'' at 60 minutes.
  \end{itemize}
\item
  E2c) At 60 minutes, each competitor must give the judge a legibly
  written solution with the competitor's name, using the notation
  defined for Outer Block Turn Metric (described in ). Penalty:
  disqualification of the attempt (DNF).
\item
  E2d) The length of the solution is calculated in Outer Block Turn
  Metric (see ).

  \begin{itemize}
  \item
    E2d1) The competitor is permitted a maximum solution length of 80
    (moves and rotations).
  \end{itemize}
\item
  E2e) The competitor's solution must not be directly derived from any
  part of the scrambling algorithm. Penalty: disqualification of the
  attempt (DNF).

  \begin{itemize}
  \item
    E2e1) The WCA Delegate may ask the competitor to explain the purpose
    of each move in his solution, irrespective of scrambling algorithm.
    If the competitor cannot give a valid explanation, the attempt is
    disqualified.
  \end{itemize}
\item
  E3) The competitor may use the following objects during the attempt.
  Penalty for using unauthorised objects: disqualification of the
  attempt (DNF).
\item
  E3a) Paper and pens (both supplied by judge).
\item
  E3b) 1-3 puzzles corresponding to the event (self-supplied).
\item
  E3c) Coloured stickers (self-supplied).
\end{itemize}

\subsection{ Article F: Clock Solving}

\begin{itemize}
\item
  F1) Standard speed solving procedures is followed, as described in
  (Speed Solving). Additional regulations that supersede the
  corresponding procedures in are described below.
\item
  F2) The judge places the scrambled puzzle onto the mat in a standing
  position.
\item
  F3) At the end of the inspection period, the competitor places the
  puzzle onto the mat in a standing position. He must not change the
  positions of any pins from their scrambled positions before the
  beginning of the solve. Penalty: disqualification (DNF).
\end{itemize}

\subsection{ Article H: Multiple Blindfolded Solving}

\begin{itemize}
\item
  H1) Standard speed solving procedures is followed, as described in
  (Blindfolded Solving). Additional regulations that supersede the
  corresponding procedures in are described below.
\item
  H1a) Before an attempt, the competitor must notify the judge of the
  number of puzzles he wishes to attempt blindfolded. The number of
  puzzles must be at least 2.

  \begin{itemize}
  \item
    H1a1) A competitor is not permitted to change the number of puzzles
    after the beginning of the attempt.
  \item
    H1a2) The organisation team must not disclose the competitor's
    requested number of puzzles until the beginning of the attempt.
  \end{itemize}
\item
  H1b) If he is attempting fewer than 6 puzzles, the competitor is
  allotted a time limit of 10 minutes times the number of puzzles in the
  attempt, else the time limit is 60 minutes.

  \begin{itemize}
  \item
    H1b1) The competitor may signal the end of the attempt at any time.
    If and when the time limit is reached, the judge stops the attempt
    and the attempt is then scored; the time limit for the attempt
    counts as the original recorded time.
  \end{itemize}
\item
  H1d) Time penalties for the puzzles of the attempt are cumulative.
\end{itemize}

\subsection{ Article Z: Optional Regulations}

Organisation teams may adopt optional regulations to facilitate the
administration of the competition. The WCA Board must approve any
optional regulations for a competition.

\begin{itemize}
\item
  Z1) The organisation team may require competitors to submit puzzles
  during registration.
\item
  Z2) The organisation team may limit the number of events per
  competitor.
\item
  Z3) The organisation team may select competitors who directly qualify
  for certain rounds of certain events based on the results of specific
  previous competitions.
\item
  Z4) The organisation team may limit the number of competitors per
  event, on either a ``first come first serve'' basis or based upon
  qualification times or rankings in the WCA world rankings of a
  previously announced calendar date.
\item
  Z5) The organisation team may prohibit competitors from participating
  in specific combinations of events.
\end{itemize}
\section{ WCA Guidelines 2013}

{[}Version: January 1, 2013{]}

\subsection{Notes}

\subsubsection{WCA Regulations}

The WCA Guidelines supplement the . Please see the Regulations for more
information about the WCA.

\subsubsection{Numbering}

Guidelines are numbered in correspondence with related regulations. Note
that mutiple Guidelines may correspond to the same Regulation, and some
Guidelines correspond to Regulations that do not exist anymore.

\subsubsection{Labels}

To be more informative, each Guideline is classified using one of the
following labels. Note that this should be treated as metadata, not as a
description of importance.

\begin{itemize}
\item
  {[}ADDITION{]} Additional information to supplement the Regulations.
\item
  {[}CLARIFICATION{]} Information to address any possible questions
  about interpretation of the Regulations.
\item
  {[}EXPLANATION{]} Information that clarifies the intent of
  Regulations.
\item
  {[}RECOMMENDATION{]} Something that is not strictly mandatory, but
  that should be done if possible.
\item
  {[}REMINDER{]} Information that may be addressed other
  Regulations/Guideline, but whose relevance is worth reiterating.
\end{itemize}

\subsection{Contents}

\subsection{ Article 1: Officials}

\begin{itemize}
\item
  1c3) {[}RECOMMENDATION{]} Results should be ready at the end of the
  last day of competition.
\item
  1c3b) {[}CLARIFICATION{]} If there are multiple groups, it is not
  necessary to identify which competitor was in which groups.
\item
  1c4) {[}RECOMMENDATION{]} Corrections to the results should be
  available within one week of the competition date.
\item
  1c10) {[}CLARIFICATION{]} It is sufficient to ensure access to a
  digital copy of the Regulations.
\item
  1h) {[}RECOMMENDATION{]} Competitors in the same group should use the
  same scramble sequences. Different groups should use different
  scramble sequences.
\item
  1h) {[}RECOMMENDATION{]} All final rounds of all events, as well as
  all Fewest Moves Solving rounds, should have the same scrambles for
  all competitors (i.e.~only 1 group).
\item
  1h1) {[}CLARIFICATION{]} Scramblers/judges should only scramble
  for/judge other competitors in the same group only if it is important
  for competition logistics.
\end{itemize}

\subsection{ Article 2: Competitors}

\begin{itemize}
\item
  2c) {[}ADDITION{]} First-time competitors should register using their
  legal name. They may register using a reasonable nickname, at the
  discretion of the WCA Delegate.
\item
  2c+) {[}ADDITION{]} Competitors must not provide intentionally
  misleading information, and returning competitors should provide
  information consistent with past information (e.g.~exact name and WCA
  ID).
\item
  2d) {[}ADDITION{]} Date of birth and contact information should be
  especially secured.
\item
  2d+) {[}RECOMMENDATION{]} If a third party (e.g.~journalist) asks the
  organisation team to be put in contact with any competitor(s), the
  competitor(s) should first be asked for consent.
\item
  2h) {[}CLARIFICATION{]} Competitors may be barefoot for Solving With
  Feet.
\item
  2j2) {[}EXAMPLE{]} For example, if a competitor is disqualified from
  an event for failing to show up for the final round, his results from
  earlier rounds remain valid.
\item
  2s) {[}REMINDER{]} Special accommodations must be noted in the
  Delegate Report.
\end{itemize}

\subsection{ Article 3: Puzzles}

\begin{itemize}
\item
  3a) {[}CLARIFICATION{]} Competitors may use puzzles of any reasonable
  size, at the discretion of the WCA Delegate.
\item
  3a+) {[}ADDITION{]} By default, a competitor should use the same
  puzzle for consecutive attempts in a speed solving round. A competitor
  may switch puzzles between attempts, at the discretion of the judge or
  WCA Delegate.
\item
  3a++) {[}CLARIFICATION{]} Competitors may borrow puzzles from other
  competitors privately, to use in competition.
\item
  3a1) {[}CLARIFICATION{]} Competitors may be disqualified if they do
  not come when they are called, or if they do not have a puzzle ready
  to submit (e.g.~if they planned to use a puzzle that another
  competitor is currently using, and therefore cannot submit their
  puzzle at the moment).
\item
  3h) {[}CLARIFICATION{]} Puzzles may be refined internally by sanding
  or lubricating.
\item
  3h+) {[}EXAMPLE{]} Examples of enhancements include: new moves are
  possible, normal moves are impossible, more pieces or faces are
  visible, colours on the backside of the puzzle are visible, moves are
  done automatically, or the puzzles has more/different solved states.
\end{itemize}

\subsection{ Article 4: Scrambling}

\begin{itemize}
\item
  4b1) {[}REMINDER{]} The WCA Delegate must never re-generate any
  scrambles to replace other ones for the purpose of filtering. For
  example, it is not be permitted to glance at the scrambles for a
  competition and generate the entire set again in order to generate
  ``fairer'' scrambles.
\item
  4b2) {[}CLARIFICATION{]} In general, all official scramble sequences
  should be kept secret during the competition and published together
  after the end of the competition (see ). In some cases (e.g.~world
  records), the organisation team may wish to release specific scrambles
  sooner after the end of a round.
\item
  4d) {[}CLARIFICATION{]} Some puzzles use standard colour schemes,
  except that black is replaced with white. In this case, black is the
  darkest colour and must not be treated as white.
\item
  4f) {[}RECOMMENDATION{]} The WCA Delegate should generate sufficient
  scrambles for the entire competition ahead of time, including spare
  scrambles for extra attempts.
\item
  4f+) {[}REMINDER{]} If the WCA Delegate generates any additional
  scrambles during the competition, he must save them (see ).
\end{itemize}

\subsection{ Article 5: Puzzle Defects}

\begin{itemize}
\item
  5b5) {[}EXAMPLE{]} Example of a piece not fully placed: a 5x5x5 centre
  piece twisted in its spot.
\item
  5b5+) {[}EXAMPLE{]} Example of judging an attempt ending with a
  defective puzzle: A Pyraminx with a detached tip may be reassembled in
  three states, one fully solved, and two ``+2'' states. An attempt
  ending in such a situation would be assigned a ``+2'' penalty.
\end{itemize}

\subsection{ Article 6: Awards/prizes/honours}

\begin{itemize}
\item
  6a) {[}SEPARATE{]}{[}ADDITION{]} Awards, prizes or honours may be
  given to competitors according to the announcement of the competition.
\item
  6b) {[}SEPARATE{]}{[}RECOMMENDATION{]} Competitors should attend the
  winner's ceremony to receive awards/prizes/honours.
\item
  6b1) {[}SEPARATE{]}{[}RECOMMENDATION{]} The winner's ceremony should
  be held in the competition venue, within one hour after the end of the
  last event.
\item
  6c) {[}SEPARATE{]}{[}RECOMMENDATION{]} Winners of awards, prizes or
  honours should be prepared to talk to journalists or any media
  covering the competition.
\item
  6d) {[}SEPARATE{]}{[}RECOMMENDATION{]} Organisation teams of
  competitions should have certificates for all category winners, signed
  by the leader of the organisation team and by the WCA delegate.
\end{itemize}

\subsection{ Article 7: Environment}

\begin{itemize}
\item
  7d) {[}SEPARATE{]}{[}ADDITION{]} The temperature of the competition
  area should be 21 to 25 degrees Celsius.
\item
  7h2) {[}SEPARATE{]}{[}ADDITION{]} The competitors in the competitors
  area should not be able to see the puzzles of the competitors on
  stage.
\end{itemize}

\subsection{ Article 8: Competitions}

\begin{itemize}
\item
  8a4) {[}RECOMMENDATION{]} Changes to increase the maximum number of
  competitors in qualification round or first round, or to add a
  qualification round for newly registered competitors should be made at
  least one month before the competition (see ).
\item
  8a4+) {[}RECOMMENDATION{]} The competition should be announced at
  least one month before the beginning of the competition.
\item
  8a5) {[}SEPARATE{]}{[}RECOMMENDATION{]} The competition should have at
  least 12 competitors.
\end{itemize}

\subsection{ Article 9: Events}

\begin{itemize}
\item
  9b) {[}ADDITION{]} The preferred format for the final of an event is
  ``Average of 5'' or ``Mean of 3'', if possible.
\item
  9b+) {[}ADDITION{]} Events other than those specified in may be held
  during a competition, but will be considered unofficial and therefore
  will not be included in the official results of the competition.
\item
  9f5) {[}CLARIFICATION{]} The result for an attempt is DNS if the
  competitor was eligible for the attempt and did not attempt it. If the
  competitor did not qualify for an attempt (e.g.~in a combined round),
  he does not have any result for the attempt.
\item
  9q) {[}SEPARATE{]}{[}RECOMMENDATION{]} Events and rounds should have
  at least 2 competitors.
\item
  9r) {[}EXPLANATION{]} The goal of a qualification round is to let
  unranked or low ranked competitors qualify for the first round of an
  event with many registered competitors.
\end{itemize}

\subsection{ Article 10: Solved State}

\begin{itemize}
\item
  10f) {[}EXPLANATION{]} The misalignment limits are selected so that
  they provide a natural cutoff between one state of a puzzle (without
  penalty) and a state one move away.
\end{itemize}

\subsection{ Article 11: Incidents}

\begin{itemize}
\item
  11e) {[}CLARIFICATION{]} Since an appeal is not guaranteed to be
  successful, the competitor may choose to keep the timer running while
  appealing it, and resume the attempt when appropriate.
\item
  11e1) {[}REMINDER{]} The extra attempt must be scrambled using an
  unmodified scramble sequence generated by an official scrambler (see ,
  ).
\end{itemize}

\subsection{ Article A: Speed Solving}

\begin{itemize}
\item
  A1a2) {[}ADDITION{]} In case of a cumulative time limit, the judge
  records the original recorded time for a DNF on the score sheet in
  parentheses, e.g. ``DNF (1:02.27)''.
\item
  A1a3) {[}REMINDER{]} The organisation team and the WCA Delegate must
  be mindful that time limits influence the strategies of the
  competitors (e.g.~rushing the first two attempts in hopes of meeting a
  cutoff in a combined round), and that changing time limits after the
  beginning of the round can disadvantage some competitors unfairly.
\item
  A1a4) {[}REMINDER{]} If a competitor has accidentally been permitted
  to exceed the time limit, the time limit must be enforced
  retroactively, and the judge, competitor, and WCA Delegate should be
  informed (see ). Judges must always be aware of the time limit for a
  current attempt (which might depend on previous attempts, in the case
  of a cumulative time limit).
\item
  A2c1) {[}CLARIFICATION{]} In the past, score cards have been used to
  cover puzzles while leaving some sides exposed. This is no longer
  permitted.
\item
  A3c3) {[}CLARIFICATION{]} Although the judge is required to reset the
  timer for the competitor (see ), the competitor may reset it
  before/during the inspection phase if the judge accidentally neglected
  to do so.
\item
  A6b) {[}EXPLANATION{]} The arbitrary value of 0.06 seconds was chosen
  to accommodate concerns about Stackmat timer malfunctions.
\item
  A6g) {[}ADDITION{]} While he is determining whether to assign a
  penalty for misalignment, the judge should not touch the puzzle.
\end{itemize}

\subsection{ Article B: Blindfolded Solving}

\begin{itemize}
\item
  B1) {[}REMINDER{]} The competitor must use a puzzle without textures,
  markings, or other features that distinguish similar pieces (see ).
  This should be given special attention for Blindfolded Solving
\item
  B1b) {[}RECOMMENDATION{]} Blindfolds should be checked by the WCA
  Delegate before use in the competition.
\end{itemize}

\subsection{ Article C: One-Handed Solving}

\begin{itemize}
\item
  C1b) {[}CLARIFICATION{]} The competitor may use both hands during
  inspection.
\item
  C1b+) {[}CLARIFICATION{]} The competitor is not required to use the
  same solving hand for different attempts of the same round.
\item
  C1b2) {[}REMINDER{]} Use of the surface is permitted while repairing
  the puzzle.
\end{itemize}

\subsection{ Article D: Solving With Feet}

\begin{itemize}
\item
  D1b) {[}CLARIFICATION{]} The competitor may wear socks while solving.
\item
  D1c) {[}REMINDER{]} While repairing puzzle defects, other body parts
  must not touch the puzzle.
\end{itemize}

\subsection{ Article E: Fewest Moves Solving}

\begin{itemize}
\item
  E2b) {[}CLARIFICATION{]} A competitor may choose to stop his attempt
  early by handing in a solution before the time limit.
\end{itemize}

\subsection{ Article H: Multiple Blindfolded Solving}

\begin{itemize}
\item
  H1b1) {[}REMINDER{]} The attempt is not disqualified for reaching the
  time limit, due to exceptions for Multiple Blindfolded Solving (see
  Regulations A1a4 and A1a5).
\item
  H1b1+) {[}ADDITION{]} The judge may permit the competitor to continue
  the attempt unofficially, but the attempt must be stopped and judged
  first, in full accordance with the Regulations.
\item
  H1d) {[}EXAMPLE{]} Example: If a competitor attempts 10 cubes, stops
  with a time of 59:57, and has two time penalties, the time for the
  result is 59:57 + 2*2 = 60:01 (also see ).
\item
  H1d) {[}EXAMPLE{]} Example: If a competitor attempts 10 cubes, and the
  judge stops him at one 60 minutes, and has two time penalties, the
  time for the result is 60:00 + 2*2 = 60:04.
\end{itemize}
